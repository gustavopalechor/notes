\documentclass[10pt,a4paper]{article}
\usepackage[utf8]{inputenc}
\usepackage{amsmath}
\usepackage{amsfonts}
\usepackage{amssymb}
\author{Jose Antonio Pérez Martín}
\usepackage{color}
\begin{document}
\section{Colores con nombre}
Probamos los siguientes colores:
\begin{itemize}
\item \textcolor{green}{Verde}
\item \textcolor{yellow}{Amarillo}
\item \textcolor{blue}{Azul}
\item \textcolor{magenta}{Magenta}
\end{itemize}
Este texto se dibuja \textcolor{red}{en rojo}
Este texto se dibuja \textcolor{cyan}{en azul raro}
\section{Colores sin nombre con RGB}
\begin{itemize}
\item \textcolor[rgb]{0,1,0}{Verde}
\item \textcolor[rgb]{1,1,0}{Amarillo}
\item \textcolor[rgb]{0,0,1}{Azul}
\item \textcolor[rgb]{0.7,0.7,0}{Amarillo chillón}
\item \textcolor[rgb]{1,0,0}{Rojo}
\item \textcolor[rgb]{0,0.5,1}{Cyan}
\end{itemize}
\section{Colores sin nombre con CMYK}
\begin{itemize}
	\item \textcolor[cmyk]{0,1,1,0}{rojo}
	\item \textcolor[cmyk]{1,0,1,0}{verde}
	\item \textcolor[cmyk]{1,1,0,0}{Azul}
\end{itemize}
\end{document}